%!TEX root = HyperspectralTextbook.tex
   
    %%%%%%%%%%%%%%%%%%%%%%%%%%%%%%%%%%%%%%%%%%%%%%%%
    %%% MACROS %%%
    
    %%%%%%%%%%%%%%%%%%%%%%%%%%%%%%%%%%%%%%%%%%%%%%%%
    %%% Demo of section head containing sample macro:
    %% To get a macro to expand correctly in a section head, with upper and
    %% lower case math, put the definition and set the box 
    %% before \begin{document}, so that when it appears in the 
    %% table of contents it will also work:
    
    \newcommand{\VT}[1]{\ensuremath{{V_{T#1}}}}
    
    
%%%%%%%%%%%%%%%%%%%%%%%%%%%%%%%%%%%%%%%%%%%%%%%%
%%% PDG MACROS %%%%%%%%%%%%%%%%%%%%%%%%%%%%%%%%%%%%%
    
    %%%%%%%%%%%%%%%%%%%
    %%% FORMATTING MACROS %%%
    
    \newcommand{\Ex}                   	{\vspace{6pt} \textbf{Example.   }}
    
    \newcommand{\SpecEx}           	{\vspace{6pt}\textbf{Spectral Example.   }}
    
    \newcommand{\Deriv}               	{\vspace{6pt}\textbf{Mathematical Derivation.   }}
    
    \newcommand{\Defn}[1]            	{\vspace{6pt}\textbf{Definition} of  \textbf{#1}.} 
    
    \newcommand{\Fact}      		{\vspace{6pt}\textbf{Fact}.}
    
    \newcommand{\Exercises}[3]    	{\begin{center}\textbf{\Large #1.\\} \vspace{6pt}
                                                           \textbf{\Large #2 Problem Set #3.}\end{center} }
                                                           
\newcommand{\Project}[3]    	{\begin{center}\textbf{\Large Project #1\\ \vspace{6pt}}
                                                           \textbf{\Large  #2\\ \vspace{6pt}}\textbf{\Large Due Date: #3.}
                                                     \end{center} }
                                                      
                                                           
    \newcommand{\Solutions}[3]    	{\textbf{\Large #1. #2 Problem Set #3.}\justify }
    
    \newcommand{\TestProblems}[2]  {\textbf{\Large #1 on #2} \vspace{24pt}\\
                                                           Name\underline{\hspace{320pt}} \\ \\ \justify}
                                                           
    \newcommand{\RedText}[1]      	{\textcolor{red}{#1}}
    \newcommand{\BlueText}[1]      	{\textcolor{blue}{#1}}
    \newcommand{\OrangeText}[1] 	{\textcolor{orange}{#1}} 
    
    \newcommand{\MyCaption}[1] {\caption{\textit{#1}}}
   
   %%%%%%%%%%%%%%%%%%%
    %%% MISC. MATH MACROS %%%
    \newcommand{\Degrees}[1]		{{#1}^{\circ}}
    \newcommand{\Set}[1]			{\mathcal{#1}}
    \newcommand{\PowerSet}[1]		{2^{\Set{{#1}}}}
    \newcommand{\Subspace}            {\sqsubset}
    \newcommand{\ClosedInt}[2]		{\left[#1,#2\right]}
    \newcommand{\OpenInt}[2]		{\left(#1,#2\right)}
    \newcommand{\NChoosem}[2]	{
       \begin{pmatrix*}
          \vspace{6pt}
         #1\\
         #2\\ 
       \end{pmatrix*}
   }
        
    %%%%%%%%%%%%%%%%%%%%%%%%%%%%%%%%%%%%%%%%%%%%%%%%
    %%%%%%%%%%%%%%%%%%%%
    %%% MACROS FOR VARIABLES %%%
    
    %%% LOWERCASE BOLD LETTERS %%%
    \newcommand{\bfa}  	{\mathbf{a}}
    \newcommand{\bfb}  	{\mathbf{b}}
    \newcommand{\bfc}  	{\mathbf{c}}
    \newcommand{\bfd}  	{\mathbf{d}}
    \newcommand{\bfe}   {\mathbf{e}}
    \newcommand{\bff}  	{\mathbf{f}}
    \newcommand{\bfg}   {\mathbf{g}}
    \newcommand{\bfh}  	{\mathbf{h}}
    \newcommand{\bfi}  	{\mathbf{i}}
    \newcommand{\bfj} 	{\mathbf{j}}
    \newcommand{\bfk}  	{\mathbf{k}}
    \newcommand{\bfl}  	{\mathbf{l}}
    \newcommand{\bfm}  	{\mathbf{m}}
    \newcommand{\bfn}  	{\mathbf{n}}
    \newcommand{\bfo}  	{\mathbf{o}}
    \newcommand{\bfp}  	{\mathbf{p}}
    \newcommand{\bfq}  	{\mathbf{q}}
    \newcommand{\bfr}   {\mathbf{r}}
    \newcommand{\bfs}  	{\mathbf{s}}
    \newcommand{\bft}   {\mathbf{t}}
    \newcommand{\bfu}  	{\mathbf{u}}
    \newcommand{\bfv}  	{\mathbf{v}}
    \newcommand{\bfw} 	{\mathbf{w}}
    \newcommand{\bfx}  	{\mathbf{x}}
    \newcommand{\bfy}  	{\mathbf{y}}
    \newcommand{\bfz}  	{\mathbf{z}}

%%% UPPERCASE BOLD LETTERS %%%
    \newcommand{\bfA}  	{\mathbf{A}}
    \newcommand{\bfB}  	{\mathbf{B}}
    \newcommand{\bfC}  	{\mathbf{C}}
    \newcommand{\bfD}  	{\mathbf{D}}
    \newcommand{\bfE}   	{\mathbf{E}}
    \newcommand{\bfF}  	{\mathbf{F}}
    \newcommand{\bfG}   	{\mathbf{G}}
    \newcommand{\bfH}  	{\mathbf{H}}
    \newcommand{\bfI}  	{\mathbf{I}}
    \newcommand{\bfJ} 	{\mathbf{J}}
    \newcommand{\bfK}  	{\mathbf{K}}
    \newcommand{\bfL}  	{\mathbf{L}}
    \newcommand{\bfM}  	{\mathbf{M}}
    \newcommand{\bfN}  	{\mathbf{N}}
    \newcommand{\bfO}  	{\mathbf{O}}
    \newcommand{\bfP}  	{\mathbf{P}}
    \newcommand{\bfQ}  	{\mathbf{Q}}
    \newcommand{\bfR}   	{\mathbf{R}}
    \newcommand{\bfS}  	{\mathbf{S}}
    \newcommand{\bfT}   	{\mathbf{T}}
    \newcommand{\bfU}  	{\mathbf{U}}
    \newcommand{\bfV}  	{\mathbf{V}}
    \newcommand{\bfW} 	{\mathbf{W}}
     \newcommand{\bfX}  	{\mathbf{X}}
     \newcommand{\bfY}  	{\mathbf{Y}}
     \newcommand{\bfZ}  	{\mathbf{Z}}
     
     %%% LOWERCASE BOLD SYMBOLS %%%
\newcommand{\bfalpha} 		{\boldsymbol{\alpha}}
\newcommand{\bfbeta} 		{\boldsymbol{\beta}}
\newcommand{\bfchi} 		{\boldsymbol{\chi}}
\newcommand{\bfdelta} 		{\boldsymbol{\delta}}
\newcommand{\bfepsilon} 	{\boldsymbol{\epsilon}}
\newcommand{\bfphi} 		{\boldsymbol{\phi}}
\newcommand{\bfgamma} 	{\boldsymbol{\gamma}}
\newcommand{\bfeta} 		{\boldsymbol{\eta}}
\newcommand{\bfiota} 		{\boldsymbol{\iota}}
\newcommand{\bfkappa} 		{\boldsymbol{\kappa}}
\newcommand{\bflambda} 	{\boldsymbol{\lambda}}
\newcommand{\bfmu} 		{\boldsymbol{\mu}}
\newcommand{\bfnu} 		{\boldsymbol{\nu}}
\newcommand{\bfomicron} 	{\boldsymbol{\omicron}}
\newcommand{\bfpi} 			{\boldsymbol{\pi}}
\newcommand{\bftheta} 		{\boldsymbol{\theta}}
\newcommand{\bfrho} 		{\boldsymbol{\rho}}
\newcommand{\bfsigma} 		{\boldsymbol{\sigma}}
\newcommand{\bftau} 		{\boldsymbol{\tau}}
\newcommand{\bfupsilon} 	{\boldsymbol{\upsilon}}
\newcommand{\bfomega} 		{\boldsymbol{\omega}}
\newcommand{\bfxi} 			{\boldsymbol{\xi}}
\newcommand{\bfpsi} 		{\boldsymbol{\psi}}
\newcommand{\bfzeta} 		{\boldsymbol{\zeta}}

     %%% UPPERCASE BOLD SYMBOLS %%%
\newcommand{\bfAlpha} 		{\boldsymbol{\Alpha}}
\newcommand{\bfBeta} 		{\boldsymbol{\Beta}}
\newcommand{\bfChi} 		{\boldsymbol{\Chi}}
\newcommand{\bfDelta} 		{\boldsymbol{\Delta}}
\newcommand{\bfEpsilon} 	{\boldsymbol{\Epsilon}}
\newcommand{\bfPhi} 		{\boldsymbol{\Phi}}
\newcommand{\bfGamma} 	{\boldsymbol{\Gamma}}
\newcommand{\bfEta} 		{\boldsymbol{\Eta}}
\newcommand{\bfIota} 		{\boldsymbol{\Iota}}
\newcommand{\bfKappa} 		{\boldsymbol{\Kappa}}
\newcommand{\bfLambda} 	{\boldsymbol{\Lambda}}
\newcommand{\bfMu} 		{\boldsymbol{\Mu}}
\newcommand{\bfNu} 		{\boldsymbol{\Nu}}
\newcommand{\bfOmicron} 	{\boldsymbol{\Omicron}}
\newcommand{\bfPi} 		{\boldsymbol{\Pi}}
\newcommand{\bfTheta} 		{\boldsymbol{\Theta}}
\newcommand{\bfRho} 		{\boldsymbol{\Rho}}
\newcommand{\bfSigma} 		{\boldsymbol{\Sigma}}
\newcommand{\bfTau} 		{\boldsymbol{\Tau}}
\newcommand{\bfUpsilon} 	{\boldsymbol{\Upsilon}}
\newcommand{\bfOmega} 	{\boldsymbol{\Omega}}
\newcommand{\bfXi} 		{\boldsymbol{\Xi}}
\newcommand{\bfPsi} 		{\boldsymbol{\Psi}}
\newcommand{\bfZeta} 		{\boldsymbol{\Zeta}}

%%% CALLIGRAPHIC UPPERCASE LETTERS %%%

\newcommand{\calL}			{\mathcal{L}}
   
    \newcommand{\bfVec}[1]{\mathbf{#1}}
       
    
    %Spectral Macros
    \newcommand{\Spectra}{\textbf{X }}
    
    %Statistical Macros
    \newcommand{\RandVec}[1]{\textbf{\textit{#1}}}
    \newcommand{\Mean}{\mathbf{\mu}}
    \newcommand{\SampMean}{\mathbf{\bar{\mu}}}
    \newcommand{\Cov}{\mathbf{C}}
    \newcommand{\SampCov}{\mathbf{\bar{C}}}
    \newcommand{\PCT}{\text{PCT}}
    \newcommand{\PCA}{\text{PCA}}
    \newcommand{\ExpVal}{\mathbb{E}}
    
    %Linear Algebra Macros
    \newcommand{\ColVec}[2]{
    \left[
    \begin{matrix} 
    \vspace{3pt}
    \; \text{          }#1_1\\
    \vspace{3pt}
    \;  \text{          }#1_2\\ 
    \vspace{3pt}
    \; \text{          }\vdots\\
    \vspace{3pt}
     \; \text{          }#1_#2
     \vspace{0pt}
    \end{matrix}
    \right]}
    
     \newcommand{\VTwoD}[2]{
    \left[
    \begin{matrix*}[r]
    \vspace{6pt}
    \; \text{          }#1\\
    \;  \text{          }#2\\ 
    \end{matrix*}
    \right]}

   
    \newcommand{\VThreeD}[3]{
    \left[
    \begin{matrix*}[r]
    \vspace{3pt}
    \; \text{          }#1\\
    \vspace{3pt}
    \;  \text{          }#2\\ 
        \vspace{3pt}
     \; \text{          }#3
     \vspace{0pt}
    \end{matrix*}
    \right]}

    
     \newcommand{\RowVec}[2]{
    \left[#1_1,#1_2,\dots,#1_#2 \right]^t }

    \newcommand{\GenMatrix}[4]{ #1=
    \begin{bmatrix}
          \vspace{6pt}
          \;  &#2_{11} & #2_{12}  & \ldots &#2_{1#4}\\
          \vspace{6pt}
           \;  &#2_{21}  & #2_{22} & \ldots &#2_{2#4}\\
           \vspace{3pt}
           \;  &\vdots & \vdots & \ddots & \vdots\\
           \vspace{3pt}
            \;  &#2_{#3 1}  &  #2_{#3 2}       &\ldots &#2_{#3 #4}
       \end{bmatrix}}
 
%%%  END CREATE GENERIC MATRIX   %%%
%%%%%%%%%%%%%%%%%%%%%%%%%    
    \newcommand{\SetOfVecs}[2]{
    \{\bfVec{#1}_1,\bfVec{#1}_2,\dots,\bfVec{#1}_{#2} \}
    }
    
    \newcommand{\norm}[1]   {\lVert\mathbf{#1}\rVert}
    \newcommand{\pnorm}[1] {\lVert\mathbf{#1}\rVert_p}
    \newcommand{\by}{\mathsf{x}}
    

%%% Algorithms

\newenvironment{Algorithm}{\fontfamily{pcr}\selectfont}{\par}
    
%%% END PDG MACROS %%%
%%%%%%%%%%%%%%%%%%%%%%%%%%%%%%%%%%%%%%%%%%%%%%%%
    
    
    %% use a box to expand the macro before we put it into the section head:
    
    \newbox\sectsavebox
    \setbox\sectsavebox=\hbox{\boldmath\VT{xyz}}
    
    %%%%%%%%%%%%%%%%% End Demo
    
    %%% END MACROS %%%
    %%%%%%%%%%%%%%%%%%%%%%%%%%%%%%%%%%%%%%%%%%%%%%%%%%%%%%%%